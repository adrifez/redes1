% Por Adrián Fernandez y Santiago Gonzalez-Carvajal-

\documentclass[11pt]{article} % Tipo de texto y tamano de letra

\usepackage[spanish]{babel}
\usepackage[utf8]{inputenc}

\begin{document}

\begin{titlepage}
\title{Memoria practica 3}
\date{Noviembre 8, 2017}
\author{Adrian Fernandez \\ Santiago Gonzalez-Carvajal}
\maketitle
\end{titlepage}

\section{Introduccion}

Con motivo de la practica 3 de Redes de Comunicaciones I, debemos utilizar scripts para analizar una traza que simula el trafico real de una red. \par
En concreto, debemos distinguir los paquetes IP y los no IP y analizar varios parámetros de los paquetes IP, como las direcciones IP destino y origen o, en el caso en el que se trate de TCP o UDP, los puertos destino y origen. \par
Una vez recopilados estos datos, debemos representarlos en ECDF's mediante scripts y explicar los resultados obtenidos.

\section{Apartado 1}
\emph{Porcentajes de paquetes (puede incluir una captura de pantalla) (punto 1 de los requisitos):\\
IP y NO IP (entendemos como NO-IP aquellos paquetes que no son ni ETH|IP ni ETH|VLAN|IP)\\
UDP, TCP, OTROS sobre los que son IP (igualmente entienda, un paquete IP como aquel que cumpla la pila ETH|IP o ETH|VLAN|IP).\\
Indique en todos los casos la expresión de filtro utilizada.}
    


\section{Apartado 2}
\emph{Top 10 de direcciones IP activas (en bytes y paquetes, y por sentido) y top 10 de puertos (en bytes y paquetes, y por sentido) (una captura de pantalla puede ser suficiente).}



\section{Apartado 3}
\emph{ECDF de los tamaños a nivel 2 de los paquetes de la traza (una por sentido, utilice la dirección MAC proporcionada por el generador).}



\section{Apartado 4}
\emph{ECDF de los tamaños a nivel 3 de los paquetes HTTP de la traza (una por sentido a nivel 4). Entenderemos como HTTP todos aquellos paquetes que usen el puerto 80 de TCP en origen o destino.}



\section{Apartado 5}
\emph{ECDF de los tamaños a nivel 3 de los paquetes DNS de la traza (una por sentido a nivel 4). Entenderemos como DNS todos aquellos paquetes que usen el puerto 53 de UDP en origen o destino.}



\section{Apartado 6}
\emph{ECDF de los tiempos entre llegadas del flujo TCP indicado por el generador de la traza (una por sentido a nivel 4).}



\section{Apartado 7}
\emph{ECDF de los tiempos entre llegadas del flujo UDP indicado por el generador de la traza (una por sentido a nivel 4).}



\section{Apartado 8}
\emph{Figura (o figuras) que muestre(n) el caudal/throughput/tasa/ancho de banda a nivel 2 en bits por segundo (b/s) y por sentido (asuma que la dirección Ethernet origen o destino es la indicada por el generador de trazas). Los segundos sin tráfico deben representarse a cero.}



\section{Apartado 9}
\emph{Todos los resultados obtenidos deben ser explicados e interpretados por los miembros de la pareja, y quedar reflejados en la memoria.}



\section{Conclusion}



\end{document}